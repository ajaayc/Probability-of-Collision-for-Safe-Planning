\documentclass[12pt]{article}
\usepackage[margin=.5in,paperwidth=8.5in,paperheight=11in,includefoot,heightrounded]{geometry}
%\usepackage{fullpage}
\usepackage{graphicx}
\usepackage{amsfonts}

%Removes References from Bibliography
\usepackage{etoolbox}
\patchcmd{\thebibliography}{\section*{\refname}}{}{}{}

\def\xavg{\frac{x_1+x_2}{2}}
\def\yavg{\frac{y_1+y_2}{2}}

\begin{document}

\begin{center}
\section*{EECS598 Final Project Proposal}
Ajaay Chandrasekaran\\
Winter 2018
\end{center}

\subsection*{Abstract}
Generation of real-time motion plans remains as an active research problem in the context of autonomous vehicles. Motion planners should prioritize generation of plans that require the least amount of time for actual execution, while satisfying differential constraints from the vehicle's dynamics. Furthermore, motion plan generation must account for natural uncertainty in a vehicle's motion in response to provided control inputs. This proposal seeks to explore robust motion planning that considers timeliness of execution for generated paths as well as probability of collision under motion uncertainty. We would explore the use of anytime motion planning algorithms in the CT-space, which continuously search for paths requiring the least amount of time to execute. Furthermore, we would attempt to plan real-time paths that yield low probabilities of collision under motion uncertainty. This planning approach would be explored via simulation of a car-like robot with second order dynamics and Gaussian motion uncertainty. We would evaluate the approach by comparing expected timeliness and probability of collision for our generated paths to those values for paths generated with general CT-Space search methods; we expect that our approach would converge more rapidly to an optimal solution--in terms of these metrics--than a method that does not consider these metrics.

\subsection*{Introduction and Background}
Real-time motion planning for autonomous vehicles continues to be actively explored by entities in both industry and academia. Motion plans for autonomous vehicles must especially be safe, as such vehicles are trusted to safely transport human life. Furthermore, plans must be executed in a timely manner; no one would want to ride an autonomous vehicle that decides to execute a 30-minute motion plan to a destination when a human driver may complete the plan in 10 minutes.

This proposal seeks to explore any-time motion planning for autonomous vehicles in the configuration-time space, or CT-Space, under the assumption of Gaussian motion uncertainty. Karaman, Walter, etc. developed an anytime algorithm$^{[1]}$ based on the RRT*$^{[2]}$, which produces an initially feasible motion plan and continuously optimizes it--given more time for computation during plan execution. Their model assumes no motion uncertainty, but the anytime nature of the algorithm should permit it to propagate motion uncertainty encountered during execution of a real-time plan. Tsai and Huang explored motion planning for robot arms by augmenting the C-Space with a time variable (CT-Space) and executing an anytime bidirectional RRT-Connect algorithm in this augmented space. Their method required them to produce an initial expected time to a goal configuration and to run the anytime algorithm to continously search for kinematically feasible paths that could execute in the optimal amount of time$^{[3]}$. Patil, Berg, and Alterovitz devised a method to estimate the probability of collision for a given motion plan for any arbitrary robot, considering uncertainty in both the motion and sensor models. They demonstrated the effectiveness of their method on a car-like mobile robot, as well as on a steerable medical needle in 3D and found it to be more accurate and faster than typical Monte-Carlo methods. This method ultimately provides an efficent means to evaluate the expected safety of a motion plan$^{[4]}$. In prior work, for example that done by Dolgov, Thrun, Montemerlo, etc., motion planning with differential constraints (e.g. for cars) often utilizes search among a set of motion primitives, which involves simulating a kinematic model with a particular control action for a small period of time in order to define the traversal from one node in the C-Space to the next node. This concept was particularly effective when utilized in motion planning for the Stanford Team's car, Junior, during the 2007 DARPA Urban Challenge.

\subsection*{Proposed Work}
This section and the following section are tentatively described. I expect that I need to more thoroughly read the work referenced above to obtain a more clear idea of the direction for this proposal.

In order to explore timely, collision-free path planning with motion uncertainty, I would implement Karaman's anytime RRT* algorithm in the CT-Space for the car-like robot with second order dynamics that is described in Patil's paper. I would determine an initial ``goal time'' via a method similar to the one described in the Tsai, Huang paper. Each time the anytime RRT* algorithm obtains a path, I would use the method in Patil's paper to compute the probability of collision for the generated path. Following that, I would use the shortcut smoothing algorithm from HW3 to optimize the path provided by the CT-Space RRT. Finally, I would effectively have both the expected time to the goal configuration and the probability of collision for the generated path from a single anytime RRT* iteration.

Given the expected time and collision probability, I think it would be interesting to propagate the car via its motion model along the planned path with some Gaussian uncertainty, and somehow feed these values as ``weights'' in a way when re-running the anytime RRT* algorithm. I would need to think further about how this could be done, but ideally it would be in a way that causes anytime RRT* to continue finding more time-efficient and lower collision-probable paths on subsequent iterations.

Most likely, low probability of collision should have a higher weight than timeliness of execution when both are considered as metrics for a path's cost. In other words, I would expect that it is more important to find the safest path than the most timely path. Both factors may be opposing objectives.

\subsection*{Proposed Experiments and Expected Outcomes}
I would evaluate my method by comparing it to the outcome of running anytime RRT* in the CT-Space under the same Gaussian motion model uncertainty assumption--but without feeding in the expected time to goal and probability of collision as weights to the next anytime RRT* iteration. I think it would be useful to evaluate how much more quickly my variant of anytime RRT* converges to a low-time, low collision probability path than the standard anytime RRT*. I would expect that since I would be somehow feeding in these metrics as weights to each new execution of anytime RRT*, my anytime RRT* would be able to converge faster to the most optimal, low-time, low collision probability path than the standard anytime RRT* in CT-space.

This evaluation seems to require a formal proof that incorporating these two metrics as weights into the next anytime RRT* iteration causes convergence. Karaman asserted that his general anytime RRT* ``almost surely converges to a solution,'' and I would need to somehow hope/assume that my modification of it still converges to a solution.

\subsection*{Weekly Schedule}
\begin{itemize}
\item \textbf{Week of March 12}
  \begin{itemize}
  \item Get feedback from Professor Berenson re this proposal
  \item Look over papers referenced in this proposal at Level 2/Level 3 reading level. Get especially familiar with anytime RRT* and CT-Space planning
  \item Figure out how to modify anytime RRT* with time and collision probability as weights on subsequent iterations (Start a backup plan if this is difficult)
  \item Look over other papers as needed
  \end{itemize}
  
\item \textbf{Week of March 19}
  \begin{itemize}
  \item Implement anytime RRT* in CT-Space as described by Karaman and Tsai for car-like robot with second order dynamics
  \end{itemize}
\item \textbf{Week of March 26}
  \begin{itemize}
  \item Finish implementing anytime RRT* in CT-Space
    \item Implement probability of collision calculation described in Patil paper
    \item Implement modified anytime RRT* algorithm using time and collision probability as weights
  \end{itemize}
\item \textbf{Week of April 2}

  \begin{itemize}
  \item Finish implementing modified anytime RRT* algorithm
    \item Compare performance of my algorithm to regular anytime RRT*
    \item Make Presentation
  \end{itemize}
\item \textbf{Week of April 9}

  \begin{itemize}
    \item Finish Presentation
    \item Write up Report
    \item Present to Class
  \end{itemize}
\end{itemize}

\subsection*{Referenced Work}

\def\bibindent{1em}
\begin{thebibliography}{99\kern\bibindent}
\makeatletter
\let\old@biblabel\@biblabel
\def\@biblabel#1{\old@biblabel{#1}\kern\bibindent}
\let\old@bibitem\bibitem
\def\bibitem#1{\old@bibitem{#1}\leavevmode\kern-\bibindent}
\makeatother
            \bibitem{K}Karaman, Sertac et al. Anytime Motion Planning using the RRT*. 2011 IEEE International Conference on Robotics and Automation (ICRA) May 9-13, 2011, Shanghai International Conference Center, Shanghai, China.
            \bibitem{K2}S. Karaman, E. Frazzoli. Incremental Sampling-based Algorithms for Optimal Motion Planning. 2010 Proceedings of Robotics: Science and Systems June 10, 2010, Zaragoza, Spain.
            \bibitem{K3}Y-C. Tsai, H-P. Huang, Motion Planning of a Dual-Arm Mobile Robot in the Configuration-Time Space, In Proc. of IEEE/RSJ Int. Conf. on Intelligent robots and systems. pp. 2458-2463, 2009. 
            \bibitem{K4}Patil, S., van den Berg, J., and Alterovitz, R. (2012). Estimating probability of collision for safe planning under gaussian motion and sensing uncertainty. In IEEE International Conference on Robotics and Automation (ICRA).
            \bibitem{K5}Dolgov, D., Thrun, S., Montemerlo, M. and Diebel, J. (2008). Practical search techniques in path planning for autonomous driving. Proceedings of the First International Symposium on Search Techniques in Artificial Intelligence and Robotics (STAIR-08), Chicago, IL. Menlo Park, CA, AAAI.

\end{thebibliography}

\end{document}
