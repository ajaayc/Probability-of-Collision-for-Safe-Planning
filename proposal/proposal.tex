\documentclass[12pt]{article}
\usepackage[margin=.5in,paperwidth=8.5in,paperheight=11in,includefoot,heightrounded]{geometry}
%\usepackage{fullpage}
\usepackage{graphicx}
\usepackage{amsfonts}

\def\xavg{\frac{x_1+x_2}{2}}
\def\yavg{\frac{y_1+y_2}{2}}

\begin{document}

\begin{center}
\section*{EECS598 Final Project Proposal}
Ajaay Chandrasekaran\\
Winter 2018
\end{center}

\subsection*{Abstract}
Generation of real-time motion plans remains as an active research problem in the context of autonomous vehicles. Motion planners should prioritize generation of plans that require the least amount of time for actual execution, while satisfying differential constraints from the vehicle's dynamics. Furthermore, motion plan generation must account for natural uncertainty in a vehicle's motion in response to provided control inputs.

This proposal seeks to explore robust motion planning that considers timeliness of execution for generated paths as well as probability of collision under motion uncertainty. We would explore the use of anytime motion planning algorithms in the CT-space, which continuously search for paths requiring the least amount of time to execute. Furthermore, we would attempt to plan real-time paths that yield low probabilities of collision under motion uncertainty. This planning approach would be explored via simulation of a car-like robot with second order dynamics and Gaussian motion uncertainty. We would evaluate the approach by comparing expected timeliness and probability of collision for our generated paths to those associated with standard C-Space search methods; we expect that our approach would be superior to other approaches when considering these values as metrics.

\subsection*{Introduction and Background}
Real-time motion planning for autonomous vehicles continues to be actively explored by entities in both industry and academia. Motion plans for autonomous vehicles must especially be safe, as such vehicles are trusted to safely transport human life. Furthermore, plans must be executed in a timely manner; no one would want to ride an autonomous vehicle that decides to execute a 30-minute motion plan to a destination when a human driver may complete the plan in 10 minutes.

This proposal seeks to explore any-time motion planning for autonomous vehicles in the configuration-time space, or CT-Space, under the assumption of Gaussian motion uncertainty. Karaman, Walter, etc. developed an anytime algorithm$^{[1]}$ based on the RRT*$^{[2]}$, which produces an initially feasible motion plan and continuously optimizes it--given more time for computation during plan execution. Their model assumes no motion uncertainty, but the anytime nature of the algorithm should permit it to propagate motion uncertainty encountered during execution of a real-time plan. Tsai and Huang explored motion planning for robot arms by augmenting the C-Space with a time variable (CT-Space) and executing an anytime bidirectional RRT-Connect algorithm in this augmented space. Their method required them to produce an initial expected time to a goal configuration and to run the anytime algorithm to continously search for kinematically feasible paths that could execute in the optimal amount of time$^{[3]}$. Patil, Berg, and Alterovitz devised a method to estimate the probability of collision for a given motion plan for any arbitrary robot, considering uncertainty in both the motion and sensor models. They demonstrated the effectiveness of their method on a car-like mobile robot, as well as on a steerable medical needle in 3D and found it to be more accurate and faster than traditional naive Monte-Carlo methods. This method ultimately provides an efficent means to evaluate the expected safety of a motion plan$^{[4]}$. In prior work, for example that done by Dolgov, Thrun, Montemerlo, etc., motion planning with differential constraints (e.g. for cars) often utilizes search among a set of motion primitives, which involves simulating a kinematic model with a particular control action for a small period of time in order to define the traversal from one node in the C-Space to the next node. This concept was particularly effective when utilized in motion planning for the Stanford Team's car, Junior, during the 2007 DARPA Urban Challenge.

\subsection*{Proposed Work}

\subsection*{Proposed Experiments and Expected Outcomes}

\subsection*{Weekly Schedule}

\subsection*{Referenced Work}
            $[1]$Karaman, Sertac et al. Anytime Motion Planning using the RRT*. 2011 IEEE International Conference on Robotics and Automation (ICRA) May 9-13, 2011, Shanghai International Conference Center, Shanghai, China\\
            $[2]$S. Karaman, E. Frazzoli. Incremental Sampling-based Algorithms for Optimal Motion Planning. 2010 Proceedings of Robotics: Science and Systems June 10, 2010, Zaragoza, Spain\\
            $[3]$Y-C. Tsai, H-P. Huang, Motion Planning of a Dual-Arm Mobile Robot in the Configuration-Time Space, In Proc. of IEEE/RSJ Int. Conf. on Intelligent robots and systems. pp. 2458-2463, 2009. \\
            $[4]$Patil, S., van den Berg, J., and Alterovitz, R. (2012). Estimating probability of collision for safe planning under gaussian motion and sensing uncertainty. In IEEE International Conference on Robotics and Automation (ICRA).\\
            $[5]$Dolgov, D., Thrun, S., Montemerlo, M. and Diebel, J. (2008). Practical search techniques in path planning for autonomous driving. Proceedings of the First International Symposium on Search Techniques in Artificial Intelligence and Robotics (STAIR-08), Chicago, IL. Menlo Park, CA, AAAI.\\

\end{document}
